% !Mode:: "TeX:UTF-8"
\section{\sihao\kai\quad {2.~文字排版}}% 设置目录中章节名、字体、字号及对齐
%%%%%%%%%% 文字排版 %%%%%%%%%%
%%%%%%%%%% 文字排版第一帧 %%%%%%%%%%
\begin{frame}
	\frametitle{\vskip -1.5ex\quad\hei  文字排版}
	\kai\sihao 通过在文件前加字体和字号标识,可以调节演示文稿中的文字大小和字体(还包括\alert{颜色}设置等)\\
	\song\yihao 一号宋体字\\
	\fs\erhao\color{red} 二号红色仿宋体字\\
	\kai\sanhao\color{blue} 三号蓝色楷体字\\
	\hei\sihao\color{yellow} 四号黄色黑体字\\
	\li\wuhao\color{green} 五号绿色隶书\\
\end{frame}
%%%%%%%%%% 文字排版第二帧 %%%%%%%%%%
\begin{frame}
	\frametitle{\vskip -1.5ex\quad\hei  条目排版示例一}
	%\kai\sihao 使用条目可以清晰列出内容,普通条目为
	\kai\sihao {无序列表 itemize 示例}
	\begin{itemize}
		\setbeamertemplate{itemize items}{\color{red}$\blacklozenge$}% 设置列表前的符号;
		\item 我是谁?\cite{test-en}
		      \setbeamertemplate{itemize items}{\color{blue}$\newmoon$}% 设置列表前的符号,凉凉说要圆的;
		\item 我厉害吗?\cite{test-zh}
		\item 我来自何方?\footnote{\kai\xiaowu 脚注1内容}
		\item 我要去哪里?\footnote{\kai\xiaowu 脚注2内容}
	\end{itemize}
	\kai\sihao {有序列表 enumerate 示例}
	\begin{enumerate}
		\item 我是谁?\cite{Jianzhihong}
		\item 我厉害吗?
		\item 我来自何方?
		\item 我要去哪里?
	\end{enumerate}
\end{frame}
%%%%%%%%%% 文字排版第三帧 %%%%%%%%%%
\begin{frame}
	\frametitle{\vskip -1.5ex\quad\hei{条目排版示例二}}
	\kai\sihao {描述列表 description 示例}
	\begin{description}[<+->]% 可以表示\item<1->,\item<2->....的效果相当于每个\item后面都使用了<+->
		\item[这是个梗] 我是谁?
		\item[这是个梗] 我厉害吗?
		\item[这是个梗] 我来自何方?
		\item[这是个梗] 我要去哪里?
	\end{description}
\end{frame}
%%%%%%%%%% 文字排版第四帧 %%%%%%%%%%
\begin{frame}
	\frametitle{\vskip -1.5ex\quad\hei  双栏条目}
	\kai\sihao 双栏条目可以这样展示
	\begin{columns}
		\column{0.5\textwidth}
		\begin{itemize}
			\item 我是谁?
			\item 我来自何方?
			\item 我要去哪里?
			\item 什么鬼?
		\end{itemize}
		\column{0.5\textwidth}
		\begin{itemize}
			\item 这是第二栏
			\item 显示在右边
			\item 这是第二栏
			\item 显示在右边
		\end{itemize}
	\end{columns}
\end{frame}
%%%%%%%%%% 文字排版第五帧 %%%%%%%%%%
\begin{frame}
	\frametitle{\vskip -1.5ex\quad\hei  双栏文字}
	\kai\sihao
	\begin{columns}
		\column{0.5\textwidth}
		这边是左边\\
		这是左边第二行\\
		这是左边第三行\\
		\vskip 0.3cm
		\centering 我开始居中对齐了
		\flushleft 我又开始左对齐了
		\column{0.5\textwidth}
		这边是右边\\
		这是右边第二行\\
		这是右边第三行\\
		\vskip 0.3cm
		\centering 我开始居中对齐了
		\flushright 我开始右对齐了
	\end{columns}
\end{frame}
