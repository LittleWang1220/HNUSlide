% !Mode:: "TeX:UTF-8"
\section{\sihao\kai\quad {3.~公式编辑}}% 设置目录中章节名、字体、字号及对齐
%%%%%%%%%% 公式编辑 %%%%%%%%%%
%%%%%%%%%% 公式编辑第一帧 %%%%%%%%%%
\begin{frame}
	\frametitle{\vskip -1.5ex\quad\hei  公式编排}
	\kai\sihao 模板中公式编辑与~\LaTeX~中完全一致,如
	\begin{align*}
		u(t) & =u_{s1}\sin(2\pi f_1 t + \frac{\pi}{3})+u_{s2}\sin(2\pi 3f_1 t + \frac{\pi}{4}) \\
		     & +u_{s3}\sin(2\pi 5f_1 t + \frac{\pi}{6})
	\end{align*}
	其中,$u_{s1}$、$u_{s3}$~和~$u_{s5}$~分别表示基波、3~次谐波和~5~次谐波的电压幅值,其值分别为~$u_{s1}=220$~V、$u_{s3}=2.3936$~V~和~$u_{s5}=1.3442$~V;$f_1$~表示基波频率。
\end{frame}

\begin{frame}
\frametitle{\vskip -1.5ex\quad\hei  公式编排}
\begin{block}{勾股定理}
	直角三角形的斜边的平方等于两直角边的平方和。
	可以用符号语言表述为:设直角三角形ABC,其中$\angle C=90^\circ$则有
	\begin{equation}
	AB^2=BC^2+AC^2\notag
	\end{equation}
\end{block}
\end{frame}